\chapter{Introduction}

\hspace{1cm} A college library management is a project that manages and stores books information electronically
according to students needs. The system helps both students and library manager to keep a constant
track of all the books available in the library. It allows both the admin and the student to search for
the desired book. It becomes necessary for colleges to keep a continuous check on the books issued
and returned and even calculate fine. This task if carried out manually will be tedious and includes
chances of mistakes. These errors are avoided by allowing the system to keep track of information
such as issue date, last date to return the book and even fine information and thus there is no need to
keep manual track of this information which thereby avoids chances of mistakes. Thus this system
reduces manual work to a great extent allows smooth flow of library activities by removing chances
of errors in the details.

\section{Key Features}
	We have developed some basic advanced key features for Library as below.
	\begin{itemize}
		\item Detailed record of Available Books in system
		\item Adding Members i.e Students, Staff Members to Library
		\item Adding Books to Library
		\item Addition of Author Details in Library
		\item Quantity of Books that is available 
		\item Issue of Books and Return of Books.
		\item Reminder to staff about Books pending towards Members
	\end{itemize}

\newpage
\section{Need for System}
The main aim of the Library Management Software is to handle the entire activity of a library. The
software keeps track of all the information about the books in the library, their cost,their complete
details and total number of books available in the Library. The user will find it easy inthis
automated system rather than using the manual writing system. The system contains a database
where all the information will be stored safely. The system is user-friendly and error free. The
“Library Management Software” has been developed to override the problems prevailing in the
practicing manual system.

